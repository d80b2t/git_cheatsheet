\documentclass[11pt,a4paper]{article}

\bibliographystyle{apalike}

\usepackage{epsfig, amsmath, natbib}
\usepackage{hyperref}

\begin{document}

\title{TITLE HERE}
\author{AUTHOR HERE}
\date{\today}
\maketitle


% Usually omit these for ApJ or MNRAS style files:
%\tableofcontents
%\listoffigures
%\listoftables

\begin{abstract}
This is a sample document which demonstrates some of the basic features
of \LaTeX.  You can easily reformat it for different document 
or bibliography styles.
\end{abstract}


\section{My github accounts}
npr247\\
d80b2t\\


\subsection{notes}
On Jul 23, 2014, at 7:37 PM, Nic Ross $<$npross@lbl.gov$>$ wrote:\\

Gang, \\

In an effort to get our code repository going, here's my github details: \\
	https://github.com/d80b2t \\
	https://github.com/d80b2t/spies\\
	https://github.com/d80b2t/group\_code\\

So, after creating a repository, you can go to that repository, click 
settings, and click "collaborators" to add collaborators. I have 
done this for John and Rob for the https://github.com/d80b2t/group\_code
repository. 

Thanks, 
Nic



\section{git Notes}

    \subsection{To Check-out:}
   {\tt  $>$ git init\\y
    $>$ git clone [url] \\
     e.g.\\ 
     $>$ git clone https://github.com/crazygirl9991/SeniorResearch\\
   }

    \subsection{Status check:}
   {\tt $>$ git status

 
    \subsection{Update your local branch:}
    {\tt $>$ git pull\\}


    \subsection{To Add new files/material:}
    {\tt $>$ git add\\
     $>$ git commit -m "[descriptive message]" } \\

   \subsection{To then place on repository:}
   {\tt $>$ git push }\\


\section{Things to learn...}
git remote add $<$name$>$ $<$ur$l>$\\
%% http://stackoverflow.com/questions/5617211/what-is-git-remote-add-and-git-push-origin-master

https://help.github.com/articles/removing-a-remote/ \\

\href{http://gitref.org/remotes/}{http://gitref.org/remotes/}\\


\section{Useful URLs}
\noindent
\href{http://stackoverflow.com/questions/5617211/what-is-git-remote-add-and-git-push-origin-master}{http://stackoverflow.com/questions/5617211/what-is-git-remote-add-and-git-push-origin-master}

\noindent
\href{https://help.github.com/articles/git-cheatsheet}{https://help.github.com/articles/git-cheatsheet/}
for the e.g. github-git-cheat-sheet.pdf \\

\noindent
\href{https://github.com/AlexZeitler/gitcheatsheet}{https://github.com/AlexZeitler/gitcheatsheet}



\end{document}

